\section{讨论}
\noindent 本次作业中仍然存在着各种不足。\\

\noindent 1、在运行mindspore的时候并不能十分完美地运行出来。神经网络的结构没有问题,在训练的时候发现出了一些问题。在训练的时候loss值一直不收敛。后来发现标签不同,loss值收敛到不同的值,并且最终的测试准确率一直在38\%,保持不变。排查原因时,我认为数据集和预处理没有问题,因为用pytorch写的网络是可以运行出来的,并且MindSpore中的训练方法是导入Model包,而后直接调用里面的model\_train函数进行训练,因此不太清楚是哪里出现问题了。另外一个重要的原因是MindSpore在Windows上不支持GPU计算,用CPU计算会很慢。\\

\noindent 2、代码的组织不好。很多时候直接在作业的文件夹下新建py文件来写一个程序,因此一方面看起来比较乱;并且很多时候导入文件或者保存文件选择的是绝对地址,因此可能不便于在别的计算机上测试。不过还好不管是数据集还是程序文件都基本上是直接在D盘中的,因此就算是绝对地址的话导入也比较方便。\\

\noindent 3、实验代码太过冗长,符号不简洁。因为实验中涉及到两种数据结构和多个指标,因此如何合理地组织符号也是一个问题。上一个问题提出的组织架构不好,一方面也是代码太过冗长的原因,有的时候写的时候没有规划好,写完了一个数据结构的相关部分后,发现另外一个不能在同一个文件中运行,因此只好另写一个文件,导致整个文件夹比较繁杂。
