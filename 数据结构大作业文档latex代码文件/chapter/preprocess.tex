\section{预处理}
为了能够进行CNN的训练,我们需要对于现有的数据进行一些预处理。预处理的文件是mydata.py,处理后的数据存放在data文件夹中,该文件夹中包括了test\_data.npy,test\_data.npy,train\_label.npy,train\_data.npy,val\_data.npy,val\_label.npy。

预处理包括了以下步骤:

\noindent1、定义获取图片路径的函数: get\_photo\_paths 函数接受一个文件夹路径作为参数,检查该文件夹是否存在,获取文件夹中的所有文件,并返回包含这些文件路径的列表。

\noindent2、定义加载训练数据的函数: get\_training\_data 函数接受两个线性表对象作为参数,用于存储数据和对应的标签。函数根据文件路径加载图像数据,将其调整大小,然后添加到数据线性表中,并根据文件路径中是否包含特定标签确定对应的标签。

\noindent3、初始化线性表和获取图片路径列表: 创建三个线性表对象 train、test、val,并通过 get\_photo\_paths 函数获取训练、测试和验证集的图像文件路径。

\noindent4、将图片信息加载到线性表中: 使用 add\_data 方法将图像信息加载到训练、测试和验证集的线性表中。

\noindent5、建立用于存放标签的线性表: 创建三个线性表对象 train\_data\_label、test\_data\_label、val\_data\_label,用于存放对应的标签信息。

\noindent6、获取训练数据: 调用 get\_training\_data 函数获取训练、测试和验证集的图像数据,并将其归一化。

\noindent7、预处理: 将图像数据的形状修改为在代码中定义的 img\_size(150)。

\noindent8、转换数据类型: 将图像数据和标签的数据类型转换为适用于 MindSpore 框架的数据类型。

\noindent9、修改标签的形状: 使用 np.squeeze 函数将标签的形状进行修改。

\noindent10、保存文件: 将处理后的图像数据和标签保存为 .npy 文件,以备后续在 MindSpore 中使用。