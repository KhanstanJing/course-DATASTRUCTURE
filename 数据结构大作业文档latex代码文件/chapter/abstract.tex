\begin{abstract}
	本次大作业主要完成了以下内容:编写两个数据结构、将待学习数据加载到数据结构中、通过神经网络学习、以及比较不同的数据结构的性能。
	
	我选择的两个数据结构是线性表和链表。我编写了这两个数据结构,并且添加了不同的成员变量和成员函数,用来实现不同的功能。
	
	由于无论是什么数据结构,其提供给神经网络用于学习的数据是一样的,而神经网络的训练轮次多,训练量大,因此我选择在训练前的数据处理部分比较不同的数据结构的优劣。评价大致分为三个任务上时间和空间共六个指标,每个指标均进行了12次实验,最后绘图比较。
	
	第一个任务为将图片信息加载到数据结构中。在这个任务中,链表在时间和空间上都不占优,链表所花时间的均值为0.002141秒,所占用内存均值为0.3197MB, 线性表所花平均时间为0.00052秒,所占用内存均值为0.0915MB,链表所花时间为线性表的4倍左右,所占用内存是线性表的3倍左右。
	
	第二个任务为将训练数据和训练标签匹配。在这个任务中,链表在时间上表现优于线性表,所花时间均值为20.782秒,而线性表所花时间为23.4787秒,链表比线性表快了12.98\%左右,但是这一优势可以看出是牺牲了空间所带来的,因为链表的占用内存均值为115.4333MB,而线性表占用内存均值为114.7891MB,不过这一差距并不明显,线性表只比链表少了0.056\%左右,因此可以设想,如果针对更加大量的数据,链表在时间上的优势完全可以弥补空间上的劣势。
	
	第三个任务为将数据转换为神经网络能够接受的类型。在这个任务中,链表依然都不占优,其花费时间的均值为0.000583,占用内存的均值为0.124MB,而线性表的所花时间均值为0.000124,占用内存均值为0.0169MB,链表花费时间约为线性表的5倍,占用的空间为线性表的7倍左右。
	
    在神经网络中,我使用了5个卷积层,前4个卷积层每个后面跟随了一个最大池化层,第2、4、5个卷积层后增加了随机失活层,最后添加了2层全连接层。除了最后一层因为是分类任务所以使用了Softmax层外,其余均使用的是ReLU激活函数。在训练过程中,采用了SGD优化器,开始设置了100个训练轮次,而后经过测试发现在175个训练轮次后训练准确率趋于稳定,测试准确率在72.5\%上下。
    
\end{abstract}

\textbf{关键词:} 深度学习,人工智能
\newpage