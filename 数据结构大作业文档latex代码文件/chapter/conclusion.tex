\section{总结}
在本次作业中,研究了线性表和链表就加载数据等方面的性能,综合比较了其在时间和空间上在各个任务中的表现优劣:将数据加载到数据结构中时由于链表要额外创建指针,因此从时间和空间上来说均比线性表表现差一些;将数据彼此匹配时,链表在时间上表现比线性表优异,分析应该是用空间换时间的原因,其相应的空间占用得更多一些;针对数据转换任务,链表在时间和空间上都并不如线性表,原因应该是链表一方面储存了指针,一方面比链表要额外经过一个转换成列表的步骤。

最后,我设计了一个神经网络,并且让神经网络学习,而后进行了不同轮次的训练,比较出训练准确度不再增加的轮次后对于准确度进行测试,完成了神经网络的学习,效果有待提升,可能是网络架构的问题,也可能是优化器等各个方面的问题。
